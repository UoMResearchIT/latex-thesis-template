% Delete the MSc option if you are doing a PhD, or replace it with MPhil
% for a Master of Philosophy thesis
%
% The 12pt option is required by the 2001/02 thesis regulations
% Last update 15th August 2007: new Abstract format and Copyright Statement

% Replace MSc with PhD for PhD theses
% Remove the twoside option for single-sided printing
\documentclass[12pt,MSc,twoside]{muthesis}
% This file is thesis.tex
% For the history of this document, see the header to the accompanying stylesheet, geogthesis.cls
% This document lays out the body of the thesis, bookends and chapter order. This is the document you need to compile.

% This section configures the key parameters of the document, as well as expanding and modifying various typesetting functions.
\usepackage[activate={true,nocompatibility},final,
						tracking=true,factor=1100,stretch=10,
						shrink=10]{microtype}															% This makes the typesetting prettier
\usepackage[T1]{tipa}																					% This package expands font options
\usepackage{amsmath, amsthm, amssymb}													% These packages extend math mode functions
\usepackage{natbib}																						% This defines how references are handled (via package natbib)
	\def\citeapos#1{\citeauthor{#1}'s (\citeyear{#1})}					% natbib settings as above
  \bibpunct{(}{)}{;}{a}{,}{,} 																% natbib settings as above, but we need to move away from Natbib
\usepackage{multicol}																					% Package allows multiple columns of text - good for some lists
\usepackage{xcolor}																						% Package allows for easier colour management

% These packages help make better tables and figures
\usepackage{tabularx}																					% This package gives the option of using tabularx rather than tabular
\usepackage{longtable} 																				% Package allows spanning of tables over multiple pages
\usepackage{multirow}																					% Package used in tables to set cell width to natural text widths
\usepackage{booktabs}																					% Allows the use of \toprule, \midrule, \bottomrule insead of \hline

% These packages help with placement of floats
\usepackage{rotating}																					% This package allows for sidewaysfigure and sideways table options
\usepackage{afterpage} 																				% Allows command \afterpage{\clearpage} to allow clever page float control
\usepackage[section]{placeins} 																% Forces a page of undrawn floats after each section c.f. \FloatBarrier
\usepackage{flafter} 																					% Forces floats to only be drawn after they appear their placement in the document
%\usepackage{samepage}																				% Enables an environment that forces elements onto the same page

% These packages modify pdf properties
\usepackage{pdflscape}																				% This package can be used to make l'scape pages l'scape in pdf docs	
\usepackage{pdfpages}																					% Package allows for clever insertion of pdf pages into a document

% These packages are all to do with automatic cross-referencing
\usepackage{footnote}																					% Provides an easy way of working with footnotes
\numberwithin{equation}{chapter}															% This resets the equation number at the start of each chapter
\setcounter{secnumdepth}{5}																		% This changes the depth of section numbering
\usepackage{caption} 																					% Allows \captionsetup[figure]{list=no} to listing on/off for floats - good for appendicies
\usepackage{subfigure} 																				% Enables the use of subfigures
%\usepackage{makeidx}																					% If you want to create an index for your thesis, you have too much time on your hands
%\usepackage{showidx}																					% Use these two packages and think about a taking up a hobby
%\usepackage[none]{hyphenat}    															% This can enable or disable hyphenation on line breaks. Why would you do this?
%\hyphenation{words}																					% This defines words that cannot be hyphenated, or defines acceptable hyphenations. Much better.

% The following are useful sundries for little jobs						
%\usepackage{siunitx}																					% Automagically generate si units
%\sisetup{detect-weight, detect-display-math}									% Settings for the above
%\sisetup{detect-inline-weight=math}													% Settings for the above

% The following options are useful for drafting documents
%\usepackage{lineno} 																					% Provides line numbers on paragraphs, and references therein. Useful for transcripts
%\usepackage{draftwatermark}																	% Prints a draft watermark
%\linenumbers 																								% Makes whole document line numbers. Remember to use lineno package and \resetlinenumber[1]

% The hyperref package is used to create clickable cross-reference links.
\usepackage[hidelinks]{hyperref}

\begin{document}
% The following changes the page colour of the pdf for dyslexics.
%\pagecolor{blue!30} 																					% Changes page colour. Useful for dyslexics.
\title{The University Thesis File}
\author{The Author's Name}
% Faculty of Life Sciences people should comment the next line out
\school{The Author's School}
\faculty{The Author's Faculty}
\def\wordcount{nnnnn}

% Uncomment the line below to suppress the `List of Tables' page (optional)
%\tablespagefalse

% Uncomment the line below to suppress the `List of Figures' page (optional)
%\figurespagefalse

% Uncomment the line below to use a customised Declaration statement
%\def\declaration{All the work in this thesis has been sourced from Google.}

\beforeabstract

Write your abstract here: Remember, it must fit on this A4 page and should
describe contents of the thesis/dissertation. Here might also be a good place
to indicate what you have achieved in the thesis/dissertation and, in the
case of a PhD, what new results you have discovered. Note that for a PhD
single-spacing is used throughout the Abstract, including displayed equations
\[
e = mc^{2}
\]
as for the above example.

\afterabstract

% The next part is optional
\prefacesection{Acknowledgements}
I would like to thank...

% The next line is NOT optional and MUST appear
\afterpreface

% The following defines the chapter files and their order.
\raggedright          																					%justifies left, according to the university specifications. Comment out for fully justified.
\chapter{Introduction}
\section{The University Thesis Document}
This document was originally served from the University of Southampton ECS pages. It has since been modified by a number of postgraduates in Geography \& Environment through the years. These folks include Chris Hackney, followed by Tom Bishop and Robin Wilson. Tom Bishop and Robin Wilson compiled the various modifications and example code into this document.

\section{How to Use This Document}
Keep a copy of this in its original form for reference --- we have tried to include a few key examples of code from our own theses. Otherwise, simply modify as appropriate.

\section{Structure}
The thesis style sheet tells \LaTeX \ how to produce the output. The thesis document is where you define the variables, for example the title, your name and the chapters. The individual chapters are usually separate files (although you may combine and split them as you see fit), and images are stored as individual files. You can modify where you store these in the thesis document. Remember that paths can be relative or absolute --- relative paths are always relative to the location of the thesis document. This is recommended as it makes everything rather more portable.
\chapter{Text and Cross-Referencing}
\section{Text}
Text can be written directly here. Sections and subsections can also be created.

\subsection{Special Characters}
Special characters can be inserted; see the crib sheets. For example, the El Ni\~no Southern Oscillation (ENSO) modulates climate across much of South America. 

\section{Cross-Referencing} \label{subsectionxrefs}
Almost any object can be cross-referenced using a unique label. For example, this is Section \ref{subsectionxrefs}. This all automatically updates so you mustn't worry about whether things point to the right place. You can do other tricks --- for example, Section \ref{subsectionxrefs} is on page \pageref{subsectionxrefs}.

\section{Inline Citations}
This document uses a package that deals with inline citations and automatically generates a bibliography. It is powerful in the way it can format the inline references; for example, it can cite a reference in parentheses \citep[\textit{e.g.}][]{Payne2011}, or without. It can cite just the author \citeauthor{Porinchu2003}, or include a figure reference \citep[\textit{e.g.}][Figure 1]{Payne2011}. 

\section{Math Mode}
\LaTeX \ allows for equations in their own environment. This can be inline, for example, $e=mc^{2}$.

\section{Quotes}
Sometimes it is nice to properly quote text, for example:

\begin{quote}						
The quick brown fox jumped over the lazy dog.					
\end{quote}

\chapter{Figures and Tables}
\section{Figures}
Figures are included using the code below. Remember there are environments for sideways figures and subfigures. 

\begin{figure}[tbp]
	\centering
		\includegraphics[width=1.00\textwidth]{example-image-a}
	\caption[Example image.]{An example images that uses the MWE package functions to call a dummy image.}
	\label{fig:editapic}
\end{figure}

\section{Tables}
Tables are complicated in \LaTeX. Table \ref{tab:itraxsetupsummary} is a simple example.

\begin{table}[tbp]
	\centering
		\begin{tabular}{lr}
		 \hline
		 Parameter     & Setting       \\ \hline
		 Tube          & Mo            \\
		 Voltage       & 30kV          \\
		 Current       & 30mA          \\
	   Exposure time & 30s           \\
		 Step size     & 400$\mu$m     \\
		 \hline			
		\end{tabular}
	\caption[Summary of parameters used in Itrax XRF data aquisition.]{Summary of parameters used in Itrax XRF data aquisition.}
	\label{tab:itraxsetupsummary}
\end{table} 

\bibliographystyle{chicago}
\bibliography{library}

% Comment the following THREE lines if you do NOT have an Appendix
%\appendix
%\chapter{A Long Proof}
%.........

% If you need more than one appendix, then just use another \chapter command
%\chapter{Yet Another Appendix}

\end{document}